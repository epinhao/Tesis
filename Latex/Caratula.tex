\documentclass[11pt]{article}

\usepackage[paperheight=23 cm, paperwidth=17 cm, top=2 cm, bottom=2 cm, left=2.5 cm, right=1.5 cm]{geometry}
\usepackage[spanish,mexico]{babel}
\usepackage{amsmath} % Some math functions
\usepackage{amsfonts} % Some math functions
\usepackage{subcaption} % Subfigures
\usepackage[utf8]{inputenc}
\usepackage{lscape} % Landscape mode
\usepackage[nottoc]{tocbibind} % Include bibliography in TOC
\usepackage{url}

\usepackage[letter,cam,center]{crop} % Crop marks
\usepackage[labelfont=bf]{caption}

\usepackage{graphicx} % OS X

%\usepackage[pdftex]{graphicx} % Windows
%\usepackage{epstopdf} % Windows

%\renewcommand{\baselinestretch}{1.6} %double-spacing

\let\EndItemize\enditemize
\def\enditemize{\EndItemize\bigskip} %space after itemize

\numberwithin{equation}{section} %equation numbers

\begin{document}

\thispagestyle{empty}
\vspace*{0.5cm}
{\centering
{\large INSTITUTO TECNOLOGICO AUTONOMO DE MEXICO}\\
\vspace{1.5cm}
\includegraphics[scale=0.60]{ITAM.eps}\\
\vspace{1.5cm}
{\Large
ESTUDIO DEL\\
LIBRO DE POSTURAS\\
DE LA\\
BOLSA MEXICANA DE VALORES\\
}
\vspace{1.5cm}
{\LARGE T E S I S}\\
{\large
QUE PARA OBTENER EL TITULO DE\\}
{\Large
INGENIERIA INDUSTRIAL\\}
{\large
PRESENTA\\}
{\Large
EMILIO RODRIGUEZ PINHAO MIESSNER\\}}
\vspace{3.5cm}

\noindent
{\large 
\begin{tabular*}{1\textwidth}{@{\extracolsep{\fill} }  l  r}
MEXICO, D.F. & 2014 \\
\end{tabular*}
}

\clearpage

\thispagestyle{empty}
\vspace*{0.5cm}
{\centering
{\large INSTITUTO TECNOLOGICO AUTONOMO DE MEXICO}\\
\vspace{1.5cm}
\includegraphics[scale=0.60]{ITAM.eps}\\
\vspace{1.5cm}
{\Large
ESTUDIO DEL\\
LIBRO DE POSTURAS\\
DE LA\\
BOLSA MEXICANA DE VALORES\\
}
\vspace{1.5cm}
{\LARGE T E S I S}\\
{\large
QUE PARA OBTENER EL TITULO DE\\}
{\Large
INGENIERIA INDUSTRIAL\\}
{\large
PRESENTA\\}
{\Large
EMILIO RODRIGUEZ PINHAO MIESSNER\\}
\vspace{1.5cm}
{\large
ASESOR: DR. MIGUEL FRANCISCO DE LASCURAIN MORHAN\\}
}
\vspace{1.5cm}

\noindent
{\large 
\begin{tabular*}{1\textwidth}{@{\extracolsep{\fill} }  l  r}
MEXICO, D.F. & 2014 \\
\end{tabular*}
}

\clearpage

\thispagestyle{empty}

{\large Con fundamento en el artículo 21 y 27 de la Ley Federal del Derecho de Autor y como titular de los derechos moral y patrimonial de la obra titulada ``ESTUDIO DEL LIBRO DE POSTURAS DE LA BOLSA MEXICANA DE VALORES'' , otorgo de manera gratuita y permanente al Instituto Tecnológico Autónomo de México y a la Biblioteca Raúl Bailléres Jr. autorización para que fijen la obra en cualquier medio, incluido el electrónico y la divulguen entre sus usuarios, profesores, estudiantes o terceras personas, sin que pueda percibir por la divulgación una contraprestación.\\

\vspace{1.5cm}
\begin{center}
EMILIO RODRIGUEZ PINHAO MIESSNER
\end{center}

\vspace{2cm}
\begin{center}
\line(1,0){250}\\
FECHA
\end{center}

\vspace{2cm}
\begin{center}
\line(1,0){250}\\
FIRMA
\end{center}

\clearpage
\thispagestyle{empty}

\section*{Agradecimientos}

\vspace{1.5cm}

A mi mamá, por su amor, ayuda y apoyo incondicional\\

A mis abuelos, por el gran apoyo en mis estudios\\

A Miguel de Lascuráin, por su gran paciencia y consejo\\

A Juan José Fernández y Ángel Kuri, por su interés en en este trabajo\\

A Cristina y Lorena, por su tiempo y ayuda en la revisión de este trabajo\\

A la Bolsa Mexicana de Valores, en especial a Jorge Alegría, Enrique Ibarra, Francisco Javier Torres Sánchez y Víctor Fernando Díaz Francés Kiesslich, por su apoyo para la obtención de la base de datos utilizada en el trabajo

\clearpage
\thispagestyle{empty}

\vspace*{5cm}
\section*{Dedicatoria}

A mi papá, con todo mi cariño

\clearpage
\thispagestyle{empty}

{\centering
{\Large
ESTUDIO DEL LIBRO DE POSTURAS\\
DE LA BOLSA MEXICANA DE VALORES\\
}
\vspace{1.5cm}
{\large
EMILIO RODRIGUEZ PINHAO MIESSNER\\
}
\vspace{1.5cm}
{\Large
\textbf{Resumen}\\
\vspace{0.5cm}
}
}
\noindent
En el trabajo se estudia por primera vez el libro de posturas de la Bolsa Mexicana de Valores. Para poder realizar el estudio, se reconstruyo el libro de posturas con los registros proporcionados por la BMV en una base de datos. Se compararon las características estadísticas de los precios, volúmenes y cancelaciones con las de otros mercados. Se propone un modelo del libro de posturas basado en procesos de Poisson. Con el modelo se trata de predecir la dirección de los movimientos del precio utilizando fracciones continuas y la transformación inversa de Laplace.\\

\vspace{1.0cm}
\noindent{\textbf{Palabras clave:} Bolsa Mexicana de Valores, Libro de Posturas, Proceso Poisson, Transformación Inversa de Laplace, Fracciones Continuas}


\end{document}
